\documentclass[uplatex]{jsarticle}
\usepackage{listings, jlisting}
\lstset{
  basicstyle={\ttfamily},
  identifierstyle={\small},
  commentstyle={\smallitshape},
  keywordstyle={\small\bfseries},
  ndkeywordstyle={\small},
  stringstyle={\small\ttfamily},
  frame={tb},
  breaklines=true,
  columns=[l]{fullflexible},
  numbers=left,
  xrightmargin=0zw,
  xleftmargin=3zw,
  numberstyle={\scriptsize},
  stepnumber=1,
  numbersep=1zw,
  lineskip=-0.5ex
}
\usepackage[dvipdfmx]{graphicx}
\title{コンピュータ科学実験3}

\author{101730153 佐治 礼仁 saji.ayahito@h.mbox.nagoya-u.ac.jp}
\date{\today}
\begin{document}
\maketitle
\newpage
\tableofcontents
\newpage
\section{実験6}
\subsection{目的}
実験5においてPL-0言語のコンパイラを実装した.実験6ではこのコンパイラの手続きを引数を用いて処理できるように拡張して,PL-1言語に対応することを目標とする.
それによって,引数について,仮引数および実引数の仕組みを習得して,正しく実装できる能力を養成することを目指す.
\subsection{実験方法}
実験5によって作成した\verb#llvm.h#および\verb#llvm.c#をLLVMコードを出力するためのCプログラムとして引き続き利用した.これは,\verb#parser.y#と連動して,プログラムをパースしながら,\verb#llvm#コードを作成するプログラムである.ソースコードを一つのファイルに記述すると利便性が落ちるためこのように分離されている.
コンパイラ中で使用する構造体の定義を\verb#llvm.h#内で行い,\verb#llvm.c#内でそれら構造体を用いて,実際の処理を行う関数を記述した.
\verb#parser.y#では,実験5に引き続いて\verb#llvm.c#の変数を\verb#extern#を用いて呼び出しながら,カプセル化を意識してプログラムを記述した.

\subsubsection{Fundecl構造体の定義}
\verb#Fundecl#構造体は,関数および手続きを定義した構造体として定義されている.
この構造体では,\verb#arity#として引数の数を,\verb#args#として仮引数を受け取って保持する.
\verb#Fundecl#の定義をソースコード\ref{lst:ex6:llvm.h:fundecl}に定義した.
\begin{lstlisting}[caption=Fundecl構造体,label=lst:ex6:llvm.h:fundecl]
/* LLVMの関数定義 */
typedef struct fundecl {
  char fname[256];      /* 関数名                      */
  unsigned arity;       /* 引数個数                    */
  Factor args[10];      /* 引数名                      */
  LLVMcode *codes;      /* 命令列の線形リストへのポインタ */
  struct fundecl *next; /* 次の関数定義へのポインタ      */
  ReturnType rettype;
} Fundecl;
\end{lstlisting}



\subsubsection{LLVMcode構造体の定義}
前述したとおりLLVMコードは,バックパッチ処理のためにLLVM命令に対応した構造体を線形リストに繋いだものをもとに出力される.
その際に使用されるLLVM命令に対応した構造体を\verb#LLVMcode#と呼称する.\verb#LLVMcode#構造体は,そのLLVMの命令の種別としてenum型の\verb#LLVMcommand#を,命令に応じた引数群\verb#args#,線形リストにするための次の命令のポインタを持つ.

\verb#LLVMcode#構造体の定義をソースコード\ref{lst:ex6:llvm.h:llvmcode}に示す.
\verb#LLVMcode#構造体は,命令の種別,命令に必要な引数,次の命令を差示すポインタをメンバ変数として持つ.命令に必要な引数は,命令によって異なるため,これらは共用体として定義して,メモリ消費を抑えるようにした.
基本的には引数は\verb#Factor#型を取るが,一部の命令はレジスタ番号としてint型を取るなどの例外がある.
ここで,関数呼び出し命令\verb#call#を,引数の個数を保持するための\verb#arity#および,実引数の\verb#Factor#を保持する\verb#args#をメンバ変数として保持するように拡張した.
なお,\verb#args#の配列の\verb#10#は,取りうる引数の最大個数を示している.

この定義をソースコード\ref{lst:ex6:llvm.h:llvmcode}に示す.

\begin{lstlisting}[caption=LLVM命令を表す構造体LLVMcode,label=lst:ex6:llvm.h:llvmcode]
typedef struct llvmcode {
  LLVMcommand command; /* 命令名 */
  union { /* 命令の引数 */
    struct { /* common_global */
      Factor retval;
    } common_global;
    struct { /* alloca */
      Factor retval;
    } alloca;
    struct { /* store  */
      Factor arg1;  Factor arg2;
    } store;
    struct { /* load   */
      Factor arg1;  Factor retval;
    } load;
    struct { /* br     */
      int arg1;
    } bruncond;
    struct { /* brc    */
      Factor arg1;  int arg2;  int arg3;
    } brcond;
    struct { /* label  */
      int l;
    } label;
    struct { /* add    */
      Factor arg1;  Factor arg2;  Factor retval;
    } add;
    struct { /* sub    */
      Factor arg1;  Factor arg2;  Factor retval;
    } sub;
    struct { /* mul    */
      Factor arg1;  Factor arg2;  Factor retval;
    } mul;
    struct { /* div    */
      Factor arg1;  Factor arg2;  Factor retval;
    } div;
    struct { /* icmp   */
      Cmptype type;  Factor arg1;  Factor arg2;  Factor retval;
    } icmp;
    struct { /* ret    */
      ReturnType type;
      Factor arg1;
    } ret;
    struct { /* printf */
      Factor retval;
      Factor arg1;
    } printf;
    struct { /* scanf */
      Factor retval;
      Factor arg1;
    } scanf;
    struct { /* call */
      Factor arg1;
      Factor args[10];
      unsigned int arity;
    } call;
  } args;
\end{lstlisting}

次に,\verb#llvm.c#中に記述した関数のうち,拡張および追加実装したものについて説明する.

\subsubsection{displayLlvmcodes}
\verb#displayLlvmcodes#関数は,\verb#LLVMcode#構造体を実際の目的コードとして出力する関数である.手続き呼び出しの際に引数を取れるように拡張するため,\verb#Call#命令に関する部分のみ変更した.
これの定義をソースコード\ref{lst:ex6:llvm.c:displayLlvmcodes}に示す.
ここでは,手続き呼び出し時に利用する実引数として保持している\verb#args#を\verb#for#文を用いて出力した.
\begin{lstlisting}[caption=displayLlvmcodes関数,label=lst:ex6:llvm.c:displayLlvmcodes]
case Call:
  fprintf(fp, "call void ");
  displayFactor(fp, (code->args).call.arg1 );
  fprintf(fp, "(");
  for (i=0;i<(code->args).call.arity;i++) {
    fprintf(fp, "i32 ");
    displayFactor(fp, (code->args).call.args[i] );
    if (i < (code->args).call.arity-1) fprintf(fp, ", ");
  }
  fprintf(fp, ")\n");
  break;
\end{lstlisting}

\subsubsection{displayLlvmfundecl関数}
\verb#displayLlvmfundecl#関数は,手続きを表示するための関数である.
この関数は,手続き・関数を表す\verb#Fundecl#のポインタおよび書き込むためのファイルポインタ\verb#fp#を受取って,そこに上記のように\verb#Fundecl#を書き出す処理を行った.

手続き・関数呼び出しでは,返り値の有無があるため返り値が存在する場合は,
\verb#define i32 @関数名 ()#と表記される.実験6の時点ではmain関数のみがこれに該当する.
一方返り値が存在しない場合は\verb#define void @関数名 ()#と表記される.

実験6では,定義された仮引数としてこれが出力されるため,次のように拡張した.

これの定義をソースコード\ref{lst:ex6:llvm.c:displayLlvmfundecl}に示す.
ここでは,保持している仮引数を\verb#for#文を用いて繰り返し表示させた.

\begin{lstlisting}[caption=displayLlvmfundecl関数の定義,label=lst:ex6:llvm.c:displayLlvmfundecl]
/* 手続きを表示する */
void displayLlvmfundecl(FILE *fp, Fundecl *decl ) {
  if (decl == NULL) return;
  unsigned int i;
  fprintf(fp, "define ");

  switch (decl->rettype) {
    case VOID:
      fprintf(fp, "void");break;
    case INT32:
      fprintf(fp, "i32");break;
  }
  fprintf(fp, " @%s(", decl->fname);

  for (i=0;i<(decl->arity);i++) {
    fprintf(fp, "i32");
    if (i < (decl->arity)-1) fprintf(fp, ", ");
  }

  fprintf(fp, ") {\n");
  displayLlvmcodes(fp, decl->codes);
  fprintf(fp, "}\n");
  if(decl->next != NULL) {
    fprintf(fp, "\n");
    displayLlvmfundecl(fp, decl->next);
  }
  return;
}
\end{lstlisting}

\subsubsection{defineCall関数}
\verb#defineCall#関数は,\verb#Call#命令である\verb#LLVMcode#をメモリ確保および作成して,\verb#pushLLVMcode#関数を呼び出してリストに連結するプログラムである.

この関数は,実験5のでは\verb#arg1#として関数名のみを受け取る関数であったが,実験6では引数を取る必要があるため,引数の個数を示す\verb#arity#および引数を示す\verb#Factor#の配列を受け取る\verb#args#を関数の引数として追加した.
また,それぞれ\verb#LLVMcode#構造体のメンバ変数としてそれらを追加するようにした.

これの定義をソースコード\ref{lst:ex6:llvm.c:defineCall}に示す.

\begin{lstlisting}[caption=defineCall関数の定義,label=lst:ex6:llvm.c:defineRet]
/* LLVM Call命令の作成 */
LLVMcode *defineCall(Factor arg1, unsigned int arity, Factor args[10]){
  LLVMcode *tmp;
  unsigned int i = 0;
  tmp = (LLVMcode *)malloc(sizeof(LLVMcode));
  tmp->next = NULL;
  tmp->command = Call;

  for (i=0;i<arity;i++) {
    (tmp->args).call.args[i] = args[arity-i-1];
  }

  (tmp->args).call.arg1 = arg1;
  (tmp->args).call.arity = arity;

  pushLLVMcode (tmp);

  return tmp;
}
\end{lstlisting}

\subsubsection{doProcedure関数}
\verb#doProcedure#関数は,\verb#main#以外の手続きおよび関数を\verb#LLVM#コードとして生成する関数である.引数を受け取るように拡張するために,\verb#proc_name#以外に引数の個数を示す\verb#arity#,仮引数名として受け取るために最長256文字の文字列型の配列\verb#args#を用意した.

引数を利用する関数では,レジスタ番号0から引数分だけ保持しておく.そして1つジャンプして,\verb#引数の数+1#より引数分のメモリを\verb#Alloca#命令で,確保する.そして\verb#store#命令で確保したメモリに実引数として受け取ったレジスタの番号の値を読み込む命令を追加する.引数は手続き内でローカル変数として扱われるため,記号表に追加するために\verb#insert#関数を呼び出す.\verb#Alloca#関数を呼び出すと\verb#cntr#をインクリメントされるため,\verb#Store#命令および\verb#insert#関数呼び出しでは\verb#cntr-1#が正しい値となる.

これの定義をソースコード\ref{lst:ex6:llvm.c:doProcedure}に示す.

\begin{lstlisting}[caption=doMainProcedure関数の定義,label=lst:ex6:llvm.c:doProcedure]
/* main以外の手続き/関数の実装 */
void doProcedure(char *proc_name, unsigned int arity, char args[10][256]) {
  // fprintf(stderr, "DEFINE PROCEDURE: %s\n", proc_name);
  codetl = NULL;
  cntr = 0;

  unsigned int i;

  Fundecl *tmp;
  tmp = (Fundecl *)malloc(sizeof(Fundecl));
  strcpy(tmp->fname, proc_name);
  tmp->arity = arity;
  tmp->rettype = VOID;

  if (decltl == NULL){
    if (declhd == NULL) {
      declhd = tmp;
    }
    decltl = tmp;
  } else {
    decltl->next = tmp;
    decltl = tmp;
  }

  cntr = arity + 1;

  for (i=0;i<arity;i++){
    defineAlloca();

    Factor arg1, arg2;
    arg1.type = LOCAL_VAR;
    arg1.val = i;

    arg2.type = LOCAL_VAR;
    arg2.val = cntr - 1;

    defineStore(arg1, arg2);

    insert(args[i], LOCAL_VAR, cntr - 1);
  }

}
\end{lstlisting}

最後に,これらの関数を\verb#parser.y#からどのように呼び出しているか説明する.

\subsubsection{構文規則内での呼び出し}

\verb#parser.y#の構文規則内で,\verb#llvm.c#の関数を呼び出しているパートや,必要な処理を行っている部分のうち,実験6で仕様を変更および実装を変えた部分を抜粋して説明する.

また,グローバル変数下に\verb#args#および\verb#arity#を定義した.

これらをソースコード\ref{lst:ex6:parser.y:rules}に示す.

\verb#proc_call#ルールでは,\verb#手続き名#または\verb#手続き名()#のいずれかで手続きを定義することができる.
まず,括弧を利用しない手続き定義方法について説明する.
この場合引数を利用しない手続き定義のため,\verb#Factor#のスタックから1つ取り出してこれを\verb#proc_name#として保持する.引数を0にして\verb#doProcedure#関数を呼び出して手続きを定義する.

次に,括弧を利用した手続き定義方法について説明する.
新たに\verb#did_list#という構文規則を定義した.この\verb#did_list#規則では,仮引数として定義された文字列を\verb#args#に追加して,\verb#arity#を正しい引数の個数に設定するように実装した.この構文規則は,還元を行う際に\verb#args#に文字列をコピーし,\verb#arity#をインクリメントした.\verb#did_list#は\verb#dummy_id_list#の略である.
よって,\verb#Factor#のスタックから1つ取り出してこれを\verb#proc_name#として保持する.引数を\verb#did_list#構文規則で正しく設定した\verb#args#および\verb#arity#を利用して,\verb#doProcedure#関数を呼び出して手続きを定義する.

\verb#proc_call_statement#ルールでも同様に\verb#手続き名#または\verb#手続き名()#のいずれかで呼び出すことができる.
まず,括弧を利用しない手続き呼び出し方法について説明する.
この場合引数を利用せずに手続きを呼び出すため\verb#Factor#スタックから1つ取り出してこれを\verb#proc_name#として保持する.引数を0にして\verb#defineCall#関数を呼び出して手続きを呼び出すようにした.

次に括弧を利用した手続き呼び出し方法について説明する.
呼び出された時点で,\verb#Factor#スタックには,引数分の\verb#Factor#が積まれている.そして,その奥には呼び出す関数名を表す\verb#Factor#が積まれている.したがって,関数名を表す\verb#Factor#が出現するまで\verb#while#構文を利用してスタックから\verb#Factor#を引数として取り出し,配列に格納する.そして,\verb#defineCall#命令を呼び出して手続きを呼び出す.

\begin{lstlisting}[caption=構文規則部の処理,label=lst:ex6:parser.y:rules]

...省略...

proc_decl
        : PROCEDURE proc_name SEMICOLON {
            Factor proc_name = factorpop();
            doProcedure(proc_name.vname, 0, args);
          } inblock {
            defineRet(VOID);
            delete();
          }
        | PROCEDURE proc_name LPAREN did_list RPAREN SEMICOLON {
            Factor f = factorpop();
            doProcedure(f.vname, arity, args);
          } inblock {
            defineRet(VOID);
            delete();
          }
        ;

proc_name
        : IDENT { insert($1, PROC_NAME, 0); }
        ;

...省略...

proc_call_statement
        : proc_call_name {
            Factor f, args[10];
            f = factorpop();
            defineCall(f, 0, args);
          }
        | proc_call_name LPAREN arg_list RPAREN {
            Factor f, args[10];
            unsigned int arity = 0;
            while (1){
              f = factorpop();
              if (f.type == PROC_NAME) break;
              else {
                args[arity] = f;
                arity ++;
              }
            }
            defineCall(f, arity, args);
          }
        ;

proc_call_name
        : IDENT{ lookup($1); }
        ;

...省略...

arg_list
        : expression
        | arg_list COMMA expression
        ;

id_list
        : IDENT { insert($1, UNDEFINED_VAR, 0); }
        | id_list COMMA IDENT { insert($3, UNDEFINED_VAR, 0); }
        ;

did_list
        : IDENT { strcpy(args[0], $1);arity = 1; }
        | did_list COMMA IDENT { strcpy(args[arity], $3);arity++; }

\end{lstlisting}

上記によって引数が正しく処理できるように実装できたので,\verb#Make#コマンドを用いてソースコードのコンパイルを行った.
そして生成された\verb#parser#プログラムを用いて,サンプルプログラムとして実験当初に与えられたプログラムをコンパイルした.
サンプルプログラムをソースコード\ref{lst:ex6:pl1a.p}に示す.

\begin{lstlisting}[caption=pl1a.p,label=lst:ex6:pl1a.p]
program PL1A;
var n, temp;
procedure fact(n);
begin
      if n <= 1 then
	 temp:=1
      else
      begin
	 fact(n - 1);
	 temp := temp * n
      end
end;
begin
   read(n);
   fact(n);
   write(temp)
end.
\end{lstlisting}

こらプログラムをソースコード\verb#lst:ex6:parser#に示されるコマンドを用いてコンパイルした.
\begin{lstlisting}[caption=parserコマンド,label=lst:ex6:parser]
./parser ./samples/pl1a.p
\end{lstlisting}

\subsection{実験結果}
コンパイルの実行結果,\verb#result.ll#に,ソースコード\ref{lst:ex6:result.ll}が出力された.ここで,\verb#define void @fact(i32)#となっているように,引数を一つ取っていることがわかった.\cite{compiler1}

\begin{lstlisting}[caption=result.ll,label=lst:ex6:result.ll]
@n = common global i32 0, align 4
@temp = common global i32 0, align 4
@.str.read = private unnamed_addr constant [3 x i8] c"%d\00", align 1
declare dso_local i32 @__isoc99_scanf(i8*, ...)
@.str.write = private unnamed_addr constant [4 x i8] c"%d\0A\00", align 1
declare dso_local i32 @printf(i8*, ...)

define void @fact(i32) {
%2 = alloca i32, align 4
store i32 %0, i32* %2, align 4
%3 = load i32, i32* %2, align 4
%4 = icmp sle i32 %3, 1
br i1 %4, label %5, label %6

; <label>:5:
store i32 1, i32* @temp, align 4
br label %12

; <label>:6:
%7 = load i32, i32* %2, align 4
%8 = sub nsw i32 %7, 1
call void @fact(i32 %8)
%9 = load i32, i32* @temp, align 4
%10 = load i32, i32* %2, align 4
%11 = mul nsw i32 %9, %10
store i32 %11, i32* @temp, align 4
br label %12

; <label>:12:
ret void
}

define i32 @main() {
%1 = alloca i32, align 4
store i32 0, i32* %1, align 4
%2 = call i32 (i8*, ...) @__isoc99_scanf(i8* getelementptr inbounds ([3 x i8], [3 x i8]* @.str.read, i64 0, i64 0), i32* @n)
%3 = load i32, i32* @n, align 4
call void @fact(i32 %3)
%4 = load i32, i32* @temp, align 4
%5 = call i32 (i8*, ...) @printf(i8* getelementptr inbounds ([4 x i8], [4 x i8]* @.str.write, i64 0, i64 0), i32 %4)
ret i32 0
}
\end{lstlisting}

次に,このLLVMコードを\verb#lli#コマンドを用いて実行したところソースコード\verb#lst:ex6:lli#ようになった.

\begin{lstlisting}[caption=parserコマンド,label=lst:ex4:lli]
$lli result.ll
7
5080
\end{lstlisting}
上記のように正しく階乗が求められていることが確かめられた.
よって,テストプログラムの実行結果が正しかったため,\verb#PL-1#言語を正しく処理できるコンパイラを実装することができた.

\subsection{考察}
実験6では,引数の数は\verb#10#までに固定する仕様で実装した.
一方一般的に利用できるプログラミング言語では引数の個数は制限されない.これを実現するためにどのように変更ができるか考える.
この言語では引数を保持する変数として配列を利用しているが,配列は最初に確保する要素する数を指定しなければならない.ゆえに動的に引数を保持するために線形リストを利用する.
実験6では手続き定義および手続き呼び出しの二箇所で引数として配列を定義している.これをソースコード\ref{lst:ex6:llvm.c:kosatu}に書き直せば良い.
\begin{lstlisting}[caption=線形リストを用いたargsリスト,label=lst:ex6:llvm.c:kosatu]
typedef struct {
  Factor val;
  Factors *next;
} Factors;
\end{lstlisting}
これは単純な線形リストであり,\verb#Factors#は次の\verb#Factors#のポインタを参照することによって\verb#head#から最後尾まで線形にたどることができる.
引数を作るたびに\verb#malloc#命令によって格納するメモリを確保し,最後尾に追加するように実装する.
これによって正しく引数を動的に増加させることができると考えた.

\section{実験7}
\subsection{目的}
実験6おいてPL-1言語のコンパイラを実装した.実験7ではこのコンパイラの手続きを関数化し,返り値を処理するように実装.
それによって返り値の仕組みを正しく習得して,正しく実装できる能力を養成することを目指す.
\subsection{実験方法}
実験6によって作成した\verb#llvm.h#および\verb#llvm.c#をLLVMコードを出力するためのCプログラムとして引き続き利用した.
コンパイラ中で使用する構造体の定義を\verb#llvm.h#内で行い,\verb#llvm.c#内でそれら構造体を用いて,実際の処理を行う関数を記述した.
\verb#parser.y#では,実験6に引き続いて\verb#llvm.c#の変数を\verb#extern#を用いて呼び出しながら,カプセル化を意識してプログラムを記述した.

\subsubsection{Fundecl構造体の定義}
\verb#Fundecl#構造体は,関数および手続きを定義した構造体として定義されている.
返り値が存在する場合と存在しない場合があるため,\verb#rettype#(\verb#VOID#または\verb#INT#のいずれか)および返り値の値\verb#retval#を関数または手続きを表す\verb#Fundecl#に保持させるようにした.
\verb#Fundecl#の定義をソースコード\ref{lst:ex7:llvm.h:fundecl}に定義した.
\begin{lstlisting}[caption=Fundecl構造体,label=lst:ex7:llvm.h:fundecl]
/* LLVMの関数定義 */
typedef struct fundecl {
  char fname[256];      /* 関数名                      */
  unsigned arity;       /* 引数個数                    */
  Factor args[10];      /* 引数名                      */
  LLVMcode *codes;      /* 命令列の線形リストへのポインタ */
  struct fundecl *next; /* 次の関数定義へのポインタ      */
  ReturnType rettype;
  Factor retval;
} Fundecl;
\end{lstlisting}

次に,\verb#llvm.c#中に記述した関数のうち,拡張および追加実装したものについて説明する.

\subsubsection{displayLlvmcodes}
\verb#displayLlvmcodes#関数は,\verb#LLVMcode#構造体を実際の目的コードとして出力する関数である.手続き呼び出しの際に引数を取れるように拡張するため,\verb#Ret#命令に関する部分のみ変更した.
また,\verb#Ret#命令は関数を示す\verb#Fundecl#が返り値および型を保持するため,\verb#displayLlvmcodes#の引数として\verb#Fundec#型のポインタ\verb#func#を取る.
これの定義をソースコード\ref{lst:ex6:llvm.c:displayLlvmcodes}に示す.
ここでは,手続き呼び出し時に利用する実引数として保持している\verb#args#を\verb#for#文を用いて出力した.
\begin{lstlisting}[caption=displayLlvmcodes関数,label=lst:ex6:llvm.c:displayLlvmcodes]

void displayLlvmcodes(FILE *fp, LLVMcode *code, Fundecl *func) {

  ...省略...

    case Ret:
      switch (func->rettype) {
        case INT32:
          fprintf(fp, "ret i32 ");
          displayFactor(fp, func->retval );
          fprintf(fp, "\n");
          break;
        case VOID:
          fprintf(fp, "ret void\n");
          break;
      }

      break;

  ...省略...

  displayLlvmcodes(fp, code->next, func);
}
\end{lstlisting}

\subsubsection{displayLlvmfundecl関数}
\verb#displayLlvmfundecl#関数は,手続きを表示するための関数である.
この関数は,手続き・関数を表す\verb#Fundecl#のポインタおよび書き込むためのファイルポインタ\verb#fp#を受取って,そこに上記のように\verb#Fundecl#を書き出す処理を行った.

手続き・関数呼び出しでは,返り値の有無があるため返り値が存在する場合は,
\verb#define i32 @関数名 ()#と表記される.実験7の時点ではmain関数のみがこれに該当する.
一方返り値が存在しない場合は\verb#define void @関数名 ()#と表記される.

実験7では,\verb#displayLlvmcodes#を呼び出す時に自分自身のポインタを引数として渡すように変更した.

これの定義をソースコード\ref{lst:ex7:llvm.c:displayLlvmfundecl}に示す.

\begin{lstlisting}[caption=displayLlvmfundecl関数の定義,label=lst:ex7:llvm.c:displayLlvmfundecl]
/* 手続きを表示する */
void displayLlvmfundecl(FILE *fp, Fundecl *decl ) {
  if (decl == NULL) return;
  unsigned int i;
  fprintf(fp, "define ");

  switch (decl->rettype) {
    case VOID:
      fprintf(fp, "void");break;
    case INT32:
      fprintf(fp, "i32");break;
  }
  fprintf(fp, " @%s(", decl->fname);

  for (i=0;i<(decl->arity);i++) {
    fprintf(fp, "i32");
    if (i < (decl->arity)-1) fprintf(fp, ", ");
  }

  fprintf(fp, ") {\n");
  displayLlvmcodes(fp, decl->codes, decl);
  fprintf(fp, "}\n");
  if(decl->next != NULL) {
    fprintf(fp, "\n");
    displayLlvmfundecl(fp, decl->next);
  }
  return;
}
\end{lstlisting}

\subsubsection{defineRet関数}
\verb#defineRet#関数は,\verb#Ret#命令である\verb#LLVMcode#をメモリ確保および作成して,\verb#pushLLVMcode#関数を呼び出してリストに連結するプログラムである.

これまでは\verb#Ret#命令で返り値を指定していたが,実験7では返り値は\verb#Fundecl#が保持するように設計したため,\verb#Ret#命令に対応する\verb#LLVMcode#は特定の値を保持しないようにした.

そのため,引数を受け取る仕様だったものを,引数を受け取らないように変更した.

これの定義をソースコード\ref{lst:ex7:llvm.c:defineRet}に示す.

\begin{lstlisting}[caption=defineCall関数の定義,label=lst:ex7:llvm.c:defineRet]
/* LLVM Ret命令の作成 */
LLVMcode *defineRet() {

  LLVMcode *tmp;
  tmp = (LLVMcode *)malloc(sizeof(LLVMcode));
  tmp->next = NULL;
  tmp->command = Ret;

  pushLLVMcode (tmp);

  return tmp;
}
\end{lstlisting}

最後に,これらの関数を\verb#parser.y#からどのように呼び出しているか説明する.

\subsubsection{構文規則内での呼び出し}

\verb#parser.y#の構文規則内で,\verb#llvm.c#の関数を呼び出しているパートや,必要な処理を行っている部分のうち,実験7で仕様を変更および実装を変えた部分を抜粋して説明する.

これらをソースコード\ref{lst:ex7:parser.y:rules}に示す.

\verb#PL-2#では新たな文法として関数定義および関数呼び出しが挙げられる.
手続きとは次の点で異なる.
\begin{itemize}
  \item {手続きでは返り値を定義することはできないが,関数は返り値として整数を利用することが可能}
  \item {手続き呼び出しは命令として実行されるが,関数呼び出しは式中で実行できる}
  \item {手続きと関数呼び出しはどちらも10以下の任意の個数の引数を取ることができる.これらは手続きおよび関数内でローカル変数と同じように利用される.}
\end{itemize}

まずは\verb#proc_decl#ルールで関数定義の構文を追加する.
ここでは基本的には手続き定義に準拠するが,\verb#decttl#すなわちその関数での返り値として\verb#Fundecl#の\verb#retval#に返り値用のレジスタ番号を保持させるようにした.
そして関数定義の最後で,\verb#Load#命令を実行し,その結果得られるレジスタ番号\verb#cntr-1#に更新しておく.
ここでレジスタ番号を更新するのは,代入する際のレジスタ番号と最後に返り値として返すレジスタ番号が異なるためである.

関数呼び出しでは,実験6で定義したのと同じように処理するようにした.
ただし文法ルールでは\verb#factor#ルール上に記述することで,式中から呼び出すように実装した.

\begin{lstlisting}[caption=構文規則部の処理,label=lst:ex7:parser.y:rules]

...省略...

outblock
        : var_decl_part subprog_decl_part { doMainProcedure(); } statement { defineRet();delete(); }
        ;

...省略...

proc_decl
        : PROCEDURE proc_name SEMICOLON {
            Factor proc_name = factorpop();
            doProcedure(proc_name.vname, 0, args, VOID);
          } inblock {
            defineRet();
            delete();
          }
        | PROCEDURE proc_name LPAREN did_list RPAREN SEMICOLON {
            Factor f = factorpop();
            doProcedure(f.vname, arity, args, VOID);
          } inblock {
            defineRet();
            delete();
          }
        | FUNCTION func_name LPAREN did_list RPAREN SEMICOLON {
            Factor f = factorpop();
            doProcedure(f.vname, arity, args, INT32);
          } statement {
            Factor arg1;
            arg1 = (decltl->retval);
            defineLoad(arg1);
            (decltl->retval).val = cntr-1;
            defineRet();
            delete();
          }
        ;

proc_name
        : IDENT { insert($1, PROC_NAME, 0, VOID); }
        ;

func_name
        : IDENT { insert($1, PROC_NAME, 0, INT32); }
        ;

...省略...

func_call
        : func_call_name LPAREN arg_list RPAREN {
            Factor f, args[10];
            unsigned int arity = 0;
            while (1){
              f = factorpop();
              if (f.type == PROC_NAME) break;
              else {
                args[arity] = f;
                arity ++;
              }
            }
            defineCall(f, arity, args, INT32);
          }
        ;

func_call_name
        : IDENT{ lookup($1); }
        ;

...省略...

factor
        : var_name
        | NUMBER { pushNumber($1); }
        | LPAREN expression RPAREN
        | func_call
        ;

...省略...

\end{lstlisting}

上記によって返り値が返ってくるように実装できたので,\verb#Make#コマンドを用いてソースコードのコンパイルを行った.
そして生成された\verb#parser#プログラムを用いて,サンプルプログラムとして実験当初に与えられたプログラムをコンパイルした.
サンプルプログラムをソースコード\ref{lst:ex7:pl2a.p},ソースコード\ref{lst:ex7:pl2b.p},ソースコード\ref{lst:ex7:pl2c.p}に示す.

\begin{lstlisting}[caption=pl2a.p,label=lst:ex7:pl2a.p]
\verb#pl2a.p#は返り値を用いた階乗の再帰関数呼び出しを行うプログラムである.
program PL2A;
var n;
function fact(n);
if n <= 0 then
   fact := 1
else
   fact := fact(n - 1) * n;
begin
   read(n);
   write(fact(n))
end.
\end{lstlisting}

\verb#pl2b.p#は累乗を計算するために再帰関数呼び出しを行うプログラムである.
引数を正しく2つ呼ぶことができるように実装されているかテストするプログラムである.
\begin{lstlisting}[caption=pl2b.p,label=lst:ex7:pl2b.p]
program PL2B;
var m, n;
function power(m,n);
if n <= 0 then
   power := 1
else
   power := power(m,n - 1) * m;
begin
   read(m);
   read(n);
   write(power(m,n))
end.
\end{lstlisting}

これらのプログラムをソースコード\verb#lst:ex7:parser#に示されるコマンドを用いてコンパイルした.
同様にして\verb#pl2b.p#もコンパイルを行った.
\begin{lstlisting}[caption=parserコマンド,label=lst:ex6:parser]
./parser ./samples/pl2a.p
\end{lstlisting}

\subsection{実験結果}
\subsubsection{pl2a.pの実行結果}
コンパイルの実行結果,\verb#result.ll#に,ソースコード\ref{lst:ex7:pl2a.p:result.ll}が出力された.
\begin{lstlisting}[caption=result.ll,label=lst:ex7:pl2a.p:result.ll]
@n = common global i32 0, align 4
@.str.read = private unnamed_addr constant [3 x i8] c"%d\00", align 1
declare dso_local i32 @__isoc99_scanf(i8*, ...)
@.str.write = private unnamed_addr constant [4 x i8] c"%d\0A\00", align 1
declare dso_local i32 @printf(i8*, ...)

define i32 @fact(i32) {
%2 = alloca i32, align 4
%3 = alloca i32, align 4
store i32 %0, i32* %3, align 4
%4 = load i32, i32* %3, align 4
%5 = icmp sle i32 %4, 0
br i1 %5, label %6, label %7

; <label>:6:
store i32 1, i32* %2, align 4
br label %13

; <label>:7:
%8 = load i32, i32* %3, align 4
%9 = sub nsw i32 %8, 1
%10 = call i32 @fact (i32 %9)
%11 = load i32, i32* %3, align 4
%12 = mul nsw i32 %10, %11
store i32 %12, i32* %2, align 4
br label %13

; <label>:13:
%14 = load i32, i32* %2, align 4
ret i32 %14
}

define i32 @main() {
%1 = alloca i32, align 4
store i32 0, i32* %1, align 4
%2 = call i32 (i8*, ...) @__isoc99_scanf(i8* getelementptr inbounds ([3 x i8], [3 x i8]* @.str.read, i64 0, i64 0), i32* @n)
%3 = load i32, i32* @n, align 4
%4 = call i32 @fact (i32 %3)
%5 = call i32 (i8*, ...) @printf(i8* getelementptr inbounds ([4 x i8], [4 x i8]* @.str.write, i64 0, i64 0), i32 %4)
ret i32 0
}
\end{lstlisting}

次に,このLLVMコードを\verb#lli#コマンドを用いて実行したところソースコード\verb#lst:ex7:pl2a.p:lli#のようになった.

\begin{lstlisting}[caption=lliコマンド,label=lst:ex7:pl2a.p:lli]
$lli result.ll
5
120
\end{lstlisting}
このプログラムは正しく階乗を求めることが確認された.

\subsubsection{pl2b.pの実行結果}
コンパイルの実行結果,\verb#result.ll#に,ソースコード\ref{lst:ex7:pl2b.p:result.ll}が出力された.
\begin{lstlisting}[caption=result.ll,label=lst:ex7:pl2b.p:result.ll]
@m = common global i32 0, align 4
@n = common global i32 0, align 4
@.str.read = private unnamed_addr constant [3 x i8] c"%d\00", align 1
declare dso_local i32 @__isoc99_scanf(i8*, ...)
@.str.write = private unnamed_addr constant [4 x i8] c"%d\0A\00", align 1
declare dso_local i32 @printf(i8*, ...)

define i32 @power(i32, i32) {
%3 = alloca i32, align 4
%4 = alloca i32, align 4
store i32 %0, i32* %4, align 4
%5 = alloca i32, align 4
store i32 %1, i32* %5, align 4
%6 = load i32, i32* %5, align 4
%7 = icmp sle i32 %6, 0
br i1 %7, label %8, label %9

; <label>:8:
store i32 1, i32* %3, align 4
br label %16

; <label>:9:
%10 = load i32, i32* %4, align 4
%11 = load i32, i32* %5, align 4
%12 = sub nsw i32 %11, 1
%13 = call i32 @power (i32 %10, i32 %12)
%14 = load i32, i32* %4, align 4
%15 = mul nsw i32 %13, %14
store i32 %15, i32* %3, align 4
br label %16

; <label>:16:
%17 = load i32, i32* %3, align 4
ret i32 %17
}

define i32 @main() {
%1 = alloca i32, align 4
store i32 0, i32* %1, align 4
%2 = call i32 (i8*, ...) @__isoc99_scanf(i8* getelementptr inbounds ([3 x i8], [3 x i8]* @.str.read, i64 0, i64 0), i32* @m)
%3 = call i32 (i8*, ...) @__isoc99_scanf(i8* getelementptr inbounds ([3 x i8], [3 x i8]* @.str.read, i64 0, i64 0), i32* @n)
%4 = load i32, i32* @m, align 4
%5 = load i32, i32* @n, align 4
%6 = call i32 @power (i32 %4, i32 %5)
%7 = call i32 (i8*, ...) @printf(i8* getelementptr inbounds ([4 x i8], [4 x i8]* @.str.write, i64 0, i64 0), i32 %6)
ret i32 0
}
\end{lstlisting}

次に,このLLVMコードを\verb#lli#コマンドを用いて実行したところソースコード\verb#lst:ex7:pl2b.p:lli#のようになった.

\begin{lstlisting}[caption=lliコマンド,label=lst:ex7:pl2b.p:lli]
$lli result.ll
3
5
243
\end{lstlisting}
このプログラムでも同様に正しく3の5乗が計算されていることが確認された.
よって,動作確認用に実行した\verb#pl2a.p#および\verb#pl2b.p#の両方共正しく動作することが確かめられた.

なお,配布された\verb#pl2c.p#は,\verb#result#関数名が重複しており,正しく動作しなかった.
このプログラムの仕様では同一の手続き名および関数名を定義することはできないため正しい動作と言える.
ただし,エラーが出力されるようにはしていなかったため,\verb#ll#ファイルが出力されてしまった.

\verb#pl2c.p#の\verb#result#を\verb#result1#および\verb#result2#に変更したところ,次のようになった.
これは1から10の和を出力するプログラムとなっている.
\begin{lstlisting}[caption=pl2c.p,label=lst:ex7:pl2c.p]
program PL2C;
var n, sum;
procedure result1(n);
   sum := sum + n;
procedure result2;
   write(sum);
begin
  n := 10;
  sum := 0;
  while n > 0 do
  begin
	  result1(n);
	  n := n - 1
  end;
  result2
end.
\end{lstlisting}

コンパイルの実行結果,\verb#result.ll#に,ソースコード\ref{lst:ex7:pl2c.p:result.ll}が出力された.
\begin{lstlisting}[caption=result.ll,label=lst:ex7:pl2c.p:result.ll]
@n = common global i32 0, align 4
@sum = common global i32 0, align 4
@.str.write = private unnamed_addr constant [4 x i8] c"%d\0A\00", align 1
declare dso_local i32 @printf(i8*, ...)

define void @result1(i32) {
%2 = alloca i32, align 4
store i32 %0, i32* %2, align 4
%3 = load i32, i32* @sum, align 4
%4 = load i32, i32* %2, align 4
%5 = add nsw i32 %3, %4
store i32 %5, i32* @sum, align 4
ret void
}

define void @result2() {
%1 = load i32, i32* @sum, align 4
%2 = call i32 (i8*, ...) @printf(i8* getelementptr inbounds ([4 x i8], [4 x i8]* @.str.write, i64 0, i64 0), i32 %1)
ret void
}

define i32 @main() {
%1 = alloca i32, align 4
store i32 0, i32* %1, align 4
store i32 10, i32* @n, align 4
store i32 0, i32* @sum, align 4
br label %2

; <label>:2:
%3 = load i32, i32* @n, align 4
%4 = icmp sgt i32 %3, 0
br i1 %4, label %5, label %9

; <label>:5:
%6 = load i32, i32* @n, align 4
call void @result1 (i32 %6)
%7 = load i32, i32* @n, align 4
%8 = sub nsw i32 %7, 1
store i32 %8, i32* @n, align 4
br label %2

; <label>:9:
call void @result2 ()
ret i32 0
}
\end{lstlisting}

次に,このLLVMコードを\verb#lli#コマンドを用いて実行したところソースコード\ref{lst:ex7:pl2c.p:lli}のようになった.

\begin{lstlisting}[caption=lliコマンド,label=lst:ex7:pl2c.p:lli]
$lli result.ll
55
\end{lstlisting}

よって,正しく動作することが確かめられた.

以上より\verb#pl2a.p#,\verb#pl2b.p#,\verb#pl2c.p#が正しく動作することが確認できたため,\verb#PL-2#言語のコンパイラの実装を完了した.

\subsection{考察}
実験7では,同じ関数名や手続き名の定義において,エラーを出力せずにLLVMコードを出力してしまうという不具合があった.
そこで,ここでは不具合をどのように修正するかを考える.
この不具合は関数定義時に,記号表で同名の関数や手続きがないことを確認する処理を挟めば良い.
そのため,記号表に関するプログラムに次の関数\verb#check_duplication#を作成する.

この関数は,手続き名\verb#proc_name#を記号表を上から順に検索して,その結果発見され,かつ手続き型\verb#PROC_NAME#である場合のみエラーを発生させ,異常終了させるプログラムである.
これによって,手続きや関数を定義する関数から呼び出すことで,正しくエラーを出力することが可能になると考えられる.

\section{実験10}
\subsection{目的}
実験10では,目的コードの最適化を行うことを目標とし,ここではリテラルの四則演算の目的コードの最適化を目標として実装する.
コンパイラにおいて目的コードの最適化は実行速度の向上につながり,最適化を正しく実装することはプログラミング言語の処理能力を上げることにつながるためである.

\subsection{実験方法}
まず,実験7が終了した時点でのコンパイラで四則演算を行い結果を出力する自作プログラム\verb#pl5a.p#をコンパイルした.
\verb#pl5a.p#プログラムをソースコード\ref{lst:ex10:pl5a.p}に示す.
\begin{lstlisting}[caption=pl5a.p,label=lst:ex10:pl5.p]
program PL5A;
begin
  write(1+3*(10-5) div 2);
end.
\end{lstlisting}

コンパイルの実行結果,\verb#result.ll#に,ソースコード\ref{lst:ex10:pl5c.p:result.ll}が出力された.
\begin{lstlisting}[caption=result.ll,label=lst:ex10:pl5c.p:result.ll]

\end{lstlisting}

\subsection{実験結果}

\subsection{考察}


\begin{thebibliography}{9}
  \bibitem{compiler1} 柏木 餅子、風薬, きつねさんでもわかるLLVM, 2013.
\end{thebibliography}

\end{document}
