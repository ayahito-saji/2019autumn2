\documentclass[uplatex]{jsarticle}
\usepackage{listings, jlisting}
\lstset{
  basicstyle={\ttfamily},
  identifierstyle={\small},
  commentstyle={\smallitshape},
  keywordstyle={\small\bfseries},
  ndkeywordstyle={\small},
  stringstyle={\small\ttfamily},
  frame={tb},
  breaklines=true,
  columns=[l]{fullflexible},
  numbers=left,
  xrightmargin=0zw,
  xleftmargin=3zw,
  numberstyle={\scriptsize},
  stepnumber=1,
  numbersep=1zw,
  lineskip=-0.5ex
}
\usepackage[dvipdfmx]{graphicx}
\title{コンピュータ科学実験3}

\author{101730153 佐治 礼仁 saji.ayahito@h.mbox.nagoya-u.ac.jp}
\date{\today}
\begin{document}
\maketitle
\section{実験1}
\subsection{目的}
この実験では,字句解析器生成器の一つである\verb#lex#を用いて,PL0言語を字句解析するCプログラム\verb#scanner.c#を生成する.
実験の目的は,\verb#lex#の使い方および,PL0言語がどのように規定されるかを正しく理解することである.コンパイラ作成において,構文解析が重要な要素であるが,字句解析は構文解析を行うための前提として不可欠である.ここでは,規定されていないトークンが入力された際に,エラーを正しく返すプログラムを目標として作成する.
\subsection{実験方法}
\begin{lstlisting}[caption=symbol.h,label=lst:ex1:symbol.h]
/*
 * symbols.h
 */

enum {
    SBEGIN = 1,                  /* begin */
    DIV,
    DO,
    ELSE,
    END,                       /* end */
    FOR,
    FORWARD,
    FUNCTION,
    IF,
    PROCEDURE,
    PROGRAM,
    READ,
    THEN,
    TO,
    VAR,
    WHILE,
    WRITE,

    PLUS,
    MINUS,
    MULT,
    EQ,                         /* = */
    NEQ,                        /* <> */
    LE,                         /* <= */
    LT,                         /* < */
    GE,                         /* >= */
    GT,                         /* > */
    LPAREN,                     /* ( */
    RPAREN,                     /* ) */
    LBRACE,                     /* [ */
    RBRACE,                     /* ] */
    COMMA,
    SEMICOLON,
    COLON,
    INTERVAL,                   /* .. */
    PERIOD,
    ASSIGN,                     /* := */
    NUMBER,
    IDENT,
};
\end{lstlisting}
\begin{lstlisting}[caption=scanner.c,label=lst:ex1:scanner.c]
%{
/*
 * scanner: scanner for PL-*
 *
 */

#include <stdio.h>
#include <string.h>
#include "symbols.h"

#define MAXLENGTH 16

typedef union {
    int num;
    char ident[MAXLENGTH+1];
} token;

/*
 * yylval という変数名にするのは,yacc との融合時にプログラムの変更を
 * 最小限にするためである.
 */
token yylval;

%}
%option yylineno
%%

begin           return SBEGIN;
div             return DIV;
do              return DO;
else            return ELSE;
end             return END;
for             return FOR;
forward         return FORWARD;
function        return FUNCTION;
if              return IF;
procedure       return PROCEDURE;
program         return PROGRAM;
read            return READ;
then            return THEN;
to              return TO;
var             return VAR;
while           return WHILE;
write           return WRITE;

"+"             return PLUS;
"-"             return MINUS;
"*"             return MULT;
"="             return EQ;
"<>"            return NEQ;
"<="            return LE;
"<"             return LT;
">="            return GE;
">"             return GT;
"("             return LPAREN;
")"             return RPAREN;
"["             return LBRACE;
"]"             return RBRACE;
","             return COMMA;
";"             return SEMICOLON;
":"             return COLON;
".."            return INTERVAL;
"."             return PERIOD;
":="            return ASSIGN;

[0-9]|[1-9][0-9]* {
    yylval.num = atoi(yytext);
    return NUMBER;
}

[a-zA-Z][0-9a-zA-Z]* {
    strcpy(yylval.ident, yytext);
    return IDENT;
}

[ \t\n] ;

. {
    fprintf(stderr, "cannot handle such characters: %s\n", yytext);
}

%%

main(int argc, char *argv[]) {
    FILE *fp;
    int tok;

    if (argc != 2) {
        fprintf(stderr, "usage: %s filename\n", argv[0]);
        exit(1);
    }

    if ((fp = fopen(argv[1], "r")) == NULL) {
        fprintf(stderr, "cannot open file: %s\n", argv[1]);
        exit(1);
    }

    /*
     * yyin は lex の内部変数であり,入力のファイルポインタを表す.
     */
    yyin = fp;

    /*
     * yylex() を呼び出すことにより,トークンが一つ切り出される.
     * yylex() の戻り値は,上のアクション部で定義した戻り値である.
     * yytext には,切り出されたトークンが文字列として格納されている.
     */
    while (tok = yylex()) {
        switch (tok) {
        case NUMBER:
            printf("\"%s\":\t%d\t%d\n", yytext, tok, yylval.num);
            break;

        case IDENT:
            printf("\"%s\":\t%d\t%s\n", yytext, tok, yylval.ident);
            break;

        default:
            printf("\"%s\":\t%d\tRESERVE\n", yytext, tok);
            break;
        }
    }
}
\end{lstlisting}
\subsection{実験結果}
\subsection{考察}

\section{実験2}
\subsection{目的}
\subsection{実験方法}
\subsection{実験結果}
\subsection{考察}


\section{実験3}
\subsection{目的}
\subsection{実験方法}
\subsection{実験結果}
\subsection{考察}

\end{document}
